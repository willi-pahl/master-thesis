Die hier besprochenen Grundlagen gehen nicht in eine Tiefe, um alle evtl. Fragen zu klären. Jedes einzelne Gebiet könnte eine Arbeit füllen. Stattdessen sollen lediglich einen kleinen Einblick geben.
\section{Künstliche Intelligenz}
KI

\subsection{Historisches}
Historie

\subsection{Maschinelles Lernen}
ML

\subsection{Lernparadigmen des ML}
Lernparadigmen

\subsubsection{Überwachtes Lernen}
überwachtes Lernen

\subsubsection{Unüberwachtes Lernen}
unüberwachtes Lernen

\subsection{Theoretische Grundlagen}
Theo. Grundlagen

%\subsubsection{Stochastik und Bayessche Verfahren}
%Stochastik

%\subsubsection{Analogismus}

%\subsubsection{Konnektionismus}

%\subsubsection{Symbolismus}

\subsection{Neuronale Netze}
KNN

\subsubsection{Neuronen im neuronalen Netz}
Neuronen

\subsubsection{Arten der neuronalen Netzen}
KNN Arten

\subsubsection{Lernprozess im neuronalen Netz / Training}
Training

\subsection{Deep Learning}
DL

\subsection{Natural Language Processing}
NPL

\section{Large Language Model}
Large Language Model

\subsection{Grundlagen}
Grundlagen

\subsubsection{Tokenisierung}
Token

\subsubsection{Embedding}
Embedding

\subsubsection{Vorhersage}
Transformer

\subsubsection{Dekodierung}
Dekodierung

\subsection{Historie der LLM}
Historie

%\subsection{Weitere Begriffe bei LLMs}
%LLMs

%\subsubsection{Halluzinationen}

\section{Orchestrierung von LLMs}
Orchestrierung

\section{Multi-Agenten-Systeme}
Multi-Agent-System

\section{Prompt Engineering}
Prompt

\section{Grundlagen bei der Entwicklung von Webanwendungen}
Webanwendung
