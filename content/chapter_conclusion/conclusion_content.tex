Es gibt Modelle welche besonders für die Webanwendungsentwicklung sehr gute Ergebnisse, auch ohne Optimierung liefern. Dennoch sollte eine Optimierung der Eingabeaufforderungen in Betracht gezogen werden, da diese mit relativ wenig Aufwand umsetzbar ist. Der Einsatz von generativer KI wird in alles Bereichen der Programmierung neue Maßstäbe setzen und nachhaltig verändern. Dieser Prozess hat bereits begonnen und immer mehr Unternehmen befassen sich mit diesem Thema und investieren in den Ausbau dieser Technologie. Dies bedeutet, dass eine Transformation auf Unternehmensebene durchzuführen ist, welche die gesamten Prozesse der Codegenerierung umfasst, auch wenn diese Transformation zu Beginn höhere Kosten verursacht.\vspace{0.2cm}

Es wird festgestellt, dass durch die Evaluation mit Benchmarks eine relevante Kenngröße zur Ermittlung der optimalen LLM gemessen werden kann, obwohl durch die Benchmarks nicht alle Aspekte der Webanwendungsentwicklung abgedeckt werden können. Ebenfalls ist es wichtig, Kenntnisse über die Auswahl geeignete Methoden in Form von Frameworks oder die Wahl der Sprache zu kennen, um die geeignete LLM auszuwählen.\vspace{0.2cm}

Alle, in dieser Arbeit vorgeschlagenen Ansätze, können langfristig zu einer Kosteneinsparung beitragen, da sich der Ressourcenverbrauch reduzieren wird. Dies wird sich bei die Entwicklungszeit für Anwendungen, für die Fehlersuche und beim Implementieren von zusätzlichen Funktionen bemerkbar machen.