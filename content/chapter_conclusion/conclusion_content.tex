Es gibt Modelle welche besonders gut für die Webanwendungsentwicklung sehr gute Ergebnisse auch ohne Optimierung liefern. Dennoch sollte eine Optimierung in Betracht gezogen werden. Der Einsatz von generativer KI wird in alles Bereichen der Programmierung neue Maßstäbe setzen und nachhaltig verändern. Dieser Prozess hat bereits begonnen und mehr und mehr Unternehmen befassen sich mit diesem Thema. Dies bedeutet, dass eine Transformation der Prozesse für die Codegenerierung auf Unternehmensebene durchzuführen ist, auch wenn diese Transformation Kosten verursacht.\vspace{0.2cm}

Es kann aber festgestellt werden das ein einheitlicher Einsatz von optimierter generativer KI die Webanwendungsentwicklung effektiver gestalten wird, da auf Daten der zurückliegenden Projekte und Informationen aus den Interaktionen mit den LLMs zugegriffen werden kann. Des Weiteren kommt es dadurch langfristig zu einer Kosteneinsparung, da sich der Ressourcenverbrauch reduzieren wird. D.h. die Entwicklungszeit für Anwendungen wird reduziert, ebenso die Zeit bei der Fehlersuche und beim Implementieren von zusätzlichen Funktionen.