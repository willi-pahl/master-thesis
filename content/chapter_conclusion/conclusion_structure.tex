\begin{tcolorbox}[
	enhanced,
	colback=red!5!white,
	colframe=red!75!black!50,
	title= Mein roter Faden
	]
	Struktur des Kapitels
	
	\begin{enumerate}
		\item \textbf{Zusammenfassung}: kurze Wiederholung der Zielsetzung d. Arbeit, Überblick der wichtigsten Ergebnisse aus Eval. und Optimierung, Fragestellung beantwortet?
		\item \textbf{Reflexion}: Stärken und Schwachen d. Arbeit, Diskussion über mögliche  Fehlerquellen, Einschätzung Optimierungsansätze oder Benchmarks
	\end{enumerate}
\end{tcolorbox}

\begin{tcolorbox}[
	enhanced,
	colback=red!5!white,
	colframe=red!75!black!50,
	title= Mein roter Faden
	]
	Unterschied Diskussion/Ausblick und Fazit\vspace{0.2cm}
	
	\begin{tabular}{|l|l|l|}
		\hline
		\textbf{Aspekt} & \textbf{Diskussion und Ausblick} & \textbf{Fazit} \\
		\hline
		\textbf{Funktion} & Kritische Analyse und & Zusammenfassung und \\
		& Zukunftsperspektive & Abschluss \\
		\hline
		\textbf{Zeitperspektive} & Zukunftsorientiert & Rückblickend \\
		\hline
		\textbf{Detaillierungsgrad} & Detailreichere Auseinandersetzung & Knapp und prägnant \\
		&  mit Ergebnissen & \\
		\hline
	\end{tabular}\vspace{0.2cm}
	
	Während \glqq Diskussion und Ausblick\grqq \ die Ergebnisse kritisch reflektiert und auf zukünftige Entwicklungen verweist, fasst das \glqq Fazit\grqq \ die Arbeit kompakt zusammen und beantwortet die Forschungsfrage. Beide Kapitel sind komplementär, aber klar voneinander zu unterscheiden.
\end{tcolorbox}
