

Wie in \cite{hartenstein_2024} beschrieben,

\textbf{Impulse für zukünftige Forschungen}\vspace{0.2cm}

Ein interessantes Feld für die Forschung ist die Nutzung generativer KI und welche Auswirkungen dies auf das menschliche Denken und Handeln hat. In der Studie \cite{chiriatti-2024} wird von einem System 0 gesprochen, welches neben den bekannten 
\begin{enumerate}
	\item System 1: schnelles,intuitives und automatisches Denken
	\item System 2: langsameres, analytisches und reflektierteres Denken
\end{enumerate}

eingeführt wird. Hierbei handelt es sich um ein Denken, welches die KI für den Menschen übernimmt. Entscheidungen und Daten werden durch die KI übernommen. Ein externes System, ähnlich wie eine USB-Festplatte eines PCs.\vspace{0.2cm}

Inwieweit können auch \textit{Small Language Models} für Programmieraufgaben eingesetzt werden. Könnte der enorme Energiebedarf und Ressourcen der LLMs durch SLMs ersetzt werden? Siehe 
\href{https://medium.com/@nageshmashette32/small-language-models-slms-305597c9edf2}{Small Language Models (SLMs)} oder \href{https://medium.com/version-1/small-but-powerful-a-deep-dive-into-small-language-models-slms-b793bdb002f2}{Small but Powerful: A Deep Dive into Small Language Models (SLMs)}. Eine weitere Forschung kann die Evaluation sein, ob Finetuned SLMs, wie Phi-2, Google Gemini Nano oder Metas Llama-2-13b bessere Ergebnisse liefern, als die LLMs.\vspace{0.2cm}

Ein weiteres Feld kann sich mit der Einführung einer KI in Firmen befassen und Fragen wie,

\begin{itemize}
	\item Wie können Entwickler bestmöglich vorbereitet werden, um die Einführung von KI reibungslos zu ermöglichen?
	\item Wie kann Datensicherheit und Datenqualität sichergestellt werden?
	\item Evaluierung von Kosten/Nutzen für die Einführung von KI in Softwareunternehmen.
\end{itemize}

evaluieren.