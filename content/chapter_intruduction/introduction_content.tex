\section{Hintergrund und Kontext}
Durch die zunehmende Globalisierung und Digitalisierung wird die Gesellschaft der Gegenwart und Zukunft geprägt. Der Ausbau von Hochgeschwindigkeitsnetze und die globale Corona-Pandemie haben diese Entwicklung noch einmal beschleunigt. Immer mehr Unternehmen erkennen die Potenziale der Digitalisierung und stellen ihre Prozesse um. Ganze Wertschöpfungsketten werden auf cloudbasierte Umgebungen umgestellt. Angefangen bei der Kommunikation, über Beschaffung und Produktion bis zum Verkauf der Waren und Dienstleistungen. In allen Stufen der Prozesse kommen webbasierte Anwendungen zum Einsatz, um die Kommunikation der Anwender mit den Systemen zu ermöglichen oder Schnittstellen für die Datenübertragung zwischen den Systemen zu gewährleisten. Durch wachsende Anzahl von Web-Anwendungen wächst auch der Druck an die Entwicklungsfirmen, ihre Anwendungen den schnell wechselnden Kundenanforderungen anzupassen (Beweis fehlt).\vspace{0.2cm}

Durch diesen Prozess getrieben, müssen Entwicklungsfirmen in immer kürzeren Release-Zyklen Softwarekomponenten hinzufügen und vorhandene erweitern. Gleichzeitig wachsen aber auch die Anforderungen an Stabilität und Sicherheit der cloudbasierten Anwendungen, sowie der Bedarf an kostengünstigeren IT-Abläufen (Beweis fehlt). Ein weiteres Problem ist der wachsende Fachkräftemangel in der Wirtschaft und die damit verbundenen steigenden Gehälter der Entwickler (Beweis fehlt).\vspace{0.2cm}

Die Verwendung künstlicher Intelligenz bei der Programmierung gewinnt immer mehr an Bedeutung. Eine Technologie die im besonderen Maße an dieser Entwicklung beteiligt ist, sind die Large Language Models. Insbesondere mit der Veröffentlichung vom ChatGPT wurde hier ein regelrechter Hype um die \acrshort{LLM}s ausgelöst. Diese Modelle erlauben eine Softwareentwicklung mit natürlicher Sprache. Tiefe Kenntnisse der verwendeten Programmiersprache sind nicht mehr in dem Maße erforderlich, wie ohne LLMs.\vspace{0.2cm}


\section{Problemstellung}
So groß der Hype um Künstliche Intelligenz auch sein mag, zurzeit kann KI noch nicht alles. Dies sollte auch bei der Verwendung von KI generiertem Inhalten und Code beachtet werden.
\epigraph[source={Vattenfall Online},etc={ KI für Unternehmen – die Grenzen der KI},author and source indent=0.5cm,dash=]{KI denkt nicht, KI trifft keine Entscheidungen. Eine KI antwortet auf eine Eingabe nicht mit der besten Antwort, sondern mit der Wahrscheinlichsten.}{Test}
Der Mensch muss die Ergebnisse prüfen, ehe generierte Programmcodestücke in vorhandene Programme eingefügt und in Produktionsumgebungen implementiert werden.\vspace{0.3cm}

Viele Entwickler setzen auf ChatGPT zur Generierung von Code, wie eine Umfrage von stackoverflow vom Mai 2024 zeigt \cite{noauthor_developers_2024}. Gleichzeitig wachsen auch die technischen Schulden bei Softwareprojekten, da diese Modelle nicht für die Entwicklung von Software optimiert sind (Beweis fehlt).\vspace{0.2cm}
%\subsection{Herausforderungen bei der Entwicklung von Webanwendungen}
%\subsection{Potenzial von LLMs in der Webentwicklung}


\section{Zielsetzung und Forschungsfragen}
Diese Arbeit soll eine Auswahl von Modellen evaluieren und dessen Brauchbarkeit für die Softwareentwicklung aufzeigen.\vspace{0.2cm}

Des Weiteren soll gezeigt werden, ob die automatisierte Verwendung beider Techniken eine Effizienz und Effektivität des Entwicklungsprozesses gesteigert werden kann.


\section{Aufbau der Arbeit}
Ein paar Worte zum Aufbau dieser Arbeit. Im Kapitel \ref{chap:state_of_research} wird der aktuelle Stand der Forschung vorgestellt und Erkenntnisse anderer Arbeiten diskutiert. Die in dieser Arbeit verwendetet Methoden, werden im Kapitel \ref{chap:methodology} behandelt. Die Implementierung der Test LLMs wird in Kapitel \ref{chap:implementation} besprochen und in Kapitel \ref{chap:evaluation} die Ergebnisse evaluiert. Bevor in Kapitel \ref{chap:conclusion} auf mögliche Folgearbeiten eingegangen wird, gibt es in Kapitel \ref{chap:application_scenarios} Anwendungsszenarien, die zu den Ergebnissen dieser Arbeit geführt haben.


\section{Abgrenzung}
Ausschluss anderer Anwendungsbereiche.

Rechtliche und ethische Überlegungen werden nur am Rande berücksichtigt.