\section{Hintergrund und Kontext}
Durch die zunehmende Globalisierung und Digitalisierung wird die Gesellschaft in der Gegenwart und Zukunft geprägt. Der Ausbau von Hochgeschwindigkeitsnetze und die globale Corona-Pandemie haben diese Entwicklung noch einmal beschleunigt. Immer mehr Unternehmen erkennen die Potenziale der Digitalisierung und passen ihre Geschäftsprozesse an und nutzen die Möglichkeiten. Ganze Wertschöpfungsketten werden auf cloudbasierte Umgebungen umgestellt. Angefangen bei der Kommunikation, über Beschaffung und Produktion bis zum Verkauf der Waren und Dienstleistungen, vergleiche mit \parencite[Seite 21 ff.]{banholzer-2020} und \cite{oswald-2022}. In allen Stufen der Prozesse kommen webbasierte Anwendungen zum Einsatz, um die Kommunikation der Anwender mit den Systemen zu ermöglichen oder Schnittstellen für die Datenübertragung zwischen den verschiedenen Systemen zu gewährleisten. Durch wachsende Anzahl von Web-Anwendungen wächst auch der Druck für die Entwicklungsfirmen, ihre Anwendungen den schnell und oft wechselnden Kundenanforderungen anzupassen.\vspace{0.2cm}

Durch diesen Prozess getrieben, müssen Entwicklungsfirmen in immer kürzeren Release-Zyklen Softwarekomponenten hinzufügen und vorhandene erweitern. Gleichzeitig wachsen aber auch die Anforderungen an Stabilität und Sicherheit der cloudbasierten Anwendungen, sowie der Bedarf an kostengünstigeren IT-Abläufen. Ein weiteres Problem ist der wachsende Fachkräftemangel in der Wirtschaft und die damit verbundenen steigenden Gehälter der Entwickler.\vspace{0.2cm}

Die Verwendung künstlicher Intelligenz bei der Programmierung gewinnt immer mehr an Bedeutung. Eine Technologie die im besonderen Maße an dieser Entwicklung beteiligt ist, sind die Large Language Models. Insbesondere mit der Veröffentlichung vom ChatGPT wurde hier ein regelrechter Hype um die \acrshort{LLM}s ausgelöst. Diese Modelle erlauben eine Softwareentwicklung mit natürlicher Sprache. Dadurch sind viele Menschen der Meinung, dass tiefere Kenntnisse in den jeweiligen Programmiersprachen nicht mehr so relevant wären, wie diese vor den LLMs waren.\vspace{0.2cm}


\section{Problemstellung}
So groß der Hype um künstliche Intelligenz auch sein mag, zurzeit kann KI nicht alle Anforderungen selbstständig lösen. Dies sollte auch bei der Verwendung von KI generierten Inhalten und Programmcodes beachtet werden.

\epigraph[
	source={Vattenfall Online},
	etc={KI für Unternehmen – die Grenzen der KI},
	author and source indent=0.5cm,
	dash=,
	after skip=0.2cm
]{KI denkt nicht, KI trifft keine Entscheidungen. Eine KI antwortet auf eine Eingabe nicht mit der besten Antwort, sondern mit der Wahrscheinlichsten.}

Der Nutzer muss die generierten Ergebnisse überprüfen, ehe erstellte Programmcodestücke in vorhandene Programme eingefügt und in Produktionsumgebungen implementiert werden. Den im Gegensatz zur natürlichen Sprache, ist bei Problemen der Codegenerierung die Syntax der jeweiligen Programmiersprache einzuhalten. Andernfalls kommt es zu Laufzeitfehlern oder einem unerwarteten Verhalten der Software.\vspace{0.2cm}

Viele Entwickler setzen auf Chatbots, wie ChatGPT oder Gemini zur Generierung von Code, wie eine Umfrage von \textit{stackoverflow} vom Mai 2024 zeigt \cite{noauthor_developers_2024}. Wenn der generierte Code ohne Prüfung und Tests in bestehende Projekte implementiert werden, kann dies dazu führen, dass o.a. technische Schulden angehäuft werden und Softwareprojekte langfristig einen hohen Wartungsaufwand benötigen.\vspace{0.2cm}

Gerade bei der Entwicklung im Bereich der Webanwendungen, welche mit PHP und JavaScript erstellt werden, wurde die Modelle nicht hinreicht getestet. Oft werden die Modelle für in Python- oder Javaproblemen evaluiert.

\section{Stand der Forschung}
In \cite{jiang-2024} wird eine bis dato fehlende Literaturrecherche zum Thema \glqq Codegenerierung durch große Sprachmodelle\grqq \ bemängelt, was in dieser Arbeit nachgeholt wird und haben im Juni 2024 Literatur zusammengetragen, welche sich mit Codegenerierung befasst.\vspace{0.2cm}

Um die Prompts im Ingenieurswesen zu optimieren, wird in \cite{velasquez-henao-2023} die GPEI (Goal Prompt Evaluation Iteration) Methodik vorgeschlagen, welche aus vier Schritten besteht. Zuerst wird das Ziel definiert, dann ein Entwurf der Anforderung, im Anschluss die Bewertung gefolgt von Iterationen.\vspace{0.2cm}

Es gibt Bestrebungen kleinere Modelle die auf Codegenerierung spezialisiert sind, mit den großen Sprachmodellen zu testen, so auch in \cite{mishra-2024}. Hier werden die Modelle als \glqq Granite Code Models\grqq -Familie zusammengefasst. Eine weitere Arbeit die sich mit kleinen Modellen, die besonders für das Generieren von Code trainiert wurden, befasst sich die Arbeit \cite{lozhkov-2024} mit StarCoder 2 betrachtet.\vspace{0.2cm}

Der wissenschaftliche Artikel \cite{nataraj-2024} befasst die sich ebenfalls mit der Web-Entwicklung mittel GPT-3. Hierbei wird die Verwendung von Generativ Adversarial Networks (GANs) vorgeschlagen, ein neuer Ansatz, mit der die Nachbearbeitung minimiert und die Codequalität optimiert wird.\vspace{0.2cm}

Eine weitere Arbeit ist \cite{zan-2022}. Diese befasst sich mit einer Umfrage zum Thema \glqq Natural Language-to-Code\grqq \ und gibt eine Übersicht über 27 Modelle und geben einen Überblick über Benchmarks und Metriken. Hier wird auch der in dieser Arbeit angewandte Benchmark \textit{HumanEval} vorgestellt.

\section{Zielsetzung und Forschungsfragen}
Das Ziel in der Softwareentwicklung war und ist die Optimierung des Entwicklungsprozesses, um Ressourcen und Kosten einzusparen und dadurch einen Wettbewerbsvorteil zu erlangen. Die steigende Nachfrage von Cloud-Anwendungen steigt auch der Optimierungsdruck in diesem Bereich besonders stark.\vspace{0.2cm}

Vor diesem Hintergrund lässt sich die Zielsetzung bereits aus dem Titel \glqq \textit{Evaluierung und Optimierung von Large Language Models für die Entwicklung von Webanwendungen}\grqq \ dieser Arbeit herleiten. Sie untersucht die Möglichkeiten mit natürlicher Sprache, Code zu generieren. In dieser Arbeit wird, wie auch in \cite[vgl. Seite 2]{jiang-2024} Language-to-Code, kurz NL2Code verwendet. Diese Arbeit soll eine Auswahl von Modellen evaluieren und dessen Brauchbarkeit für die Entwicklung von Webanwendungen aufzeigen. Um die Antworten der Modelle zu optimieren, soll eine Evaluation von Methodiken erfolgen, bei der deren Anwendung auf die Modelle eine Verbesserung der Antworten ersichtlich ist. Des Weiteren soll gezeigt werden, inwieweit sich der Prozess der Codegenerierung automatisieren lässt.\vspace{0.2cm}

Einen letzten Punkt soll der allgemeine Zustand des Codes evaluieren. Ist der Code erst einmal erstellt und muss von anderen Programmierer überarbeitet und verstanden werden, kostet dies wesentlich mehr Zeit als Code zu schreiben. Aus diesem Grund sollte Code unter anderem Kommentare enthalten und strukturiert sein.\vspace{0.2cm}

Die vier Ziele dieser Arbeit lassen sich in den folgenden kurz formulierten Sätzen zusammenfassen,

\begin{myitemize}
	\item[Z1] Welche Modelle eigenen sich für die Softwareentwicklung?
	\item[Z2] Welche Methodiken helfen die Qualität der Antworten von Modellen zu verbessern?
	\item[Z3] Wie weit lässt sich die Verwendung von Sprachmodellen, für die Codegenerierung automatisieren?
	\item[Z4] Wie gut sind die Ergebnisse, hinsichtlich Coding-Standards?
\end{myitemize}

%-------------------------------------------------------------------------------------------------


\section{Aufbau der Arbeit}
Um ein grundlegendes Verständnis zubekommen, werden im Kapitel \ref{chap:basics} die Grundlagen für diese Arbeit vorgestellt.\vspace{0.2cm}

Auf die Implementierung wird in Kapitel \ref{chap:implementation} eingegangen, welche für die Codegenerierung und Evaluierung erforderlich sind. Die daraus resultierenden Ergebnisse werden in Kapitel \ref{chap:evaluation} diskutiert.\vspace{0.2cm}

Die negativen und positiven Erfahrungen, sowie die Herausforderungen dieser Arbeit werden in Kapitel \ref{chap:lessons_learned} aufgegriffen und Lösungsansätze vorgeschlagen.\vspace{0.2cm}

Bevor in Kapitel \ref{chap:conclusion} die Arbeit zusammengefasst und ein Fazit gezogen wird, werden im Kapitel \ref{chap:discussion} die Ergebnisse erläutert, diskutiert und mögliche Impulse für künftige Arbeiten und den praktischen Einsatz angesprochen.

%-------------------------------------------------------------------------------------------------


\section{Abgrenzung}
In dieser Arbeit fokussiert sich die Betrachtung auf den Bereich der Webanwendungsentwicklung und deren verwendete Programmiersprachen. Parallelen zu anderen Anwendungsbereichen, wie beispielsweise Desktop-Anwendungsentwicklung werden hier nicht expliziert betrachtet können aber durchaus vorkommen.\vspace{0.2cm}

Auch wenn rechtliche und ethische Überlegungen einen wichtigen Aspekt in Umgang mit Künstlicher Intelligenz darstellt, wird dies in dieser Arbeit nicht betrachtet. Es gibt hinreichend Literatur zu diesen Themen, die in dieser Arbeit Beachtung finden, es wird aber nicht explizit darauf eingegangen.

