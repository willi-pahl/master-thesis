% 1. Hintergrund und Kontext
%  Einführung in das Thema: Eine kurze Darstellung des übergeordneten Themas der Arbeit, um dem Leser ein allgemeines Verständnis zu geben.
%  Relevanz und Bedeutung: Erklärung, warum das Thema wichtig ist, und seine Relevanz im akademischen, industriellen oder gesellschaftlichen Kontext.
% 2. Problemstellung
%  Beschreibung des Problems: Detaillierte Darstellung des spezifischen Problems oder der Fragestellung, die in der Arbeit behandelt wird.
%  Konkrete Herausforderungen: Aufzeigen der spezifischen Herausforderungen und Fragen, die sich aus dem Problem ergeben.
% 3. Zielsetzung und Forschungsfragen
%  Ziele der Arbeit: Klar formulierte Ziele, die mit der Arbeit erreicht werden sollen.
%  Forschungsfragen: Präzise Forschungsfragen, die die Arbeit zu beantworten versucht.
% 4. Methodik
%  Überblick über die Methoden: Kurze Beschreibung der methodischen Ansätze, die verwendet werden, um die Forschungsfragen zu beantworten.
%  Datenquellen und Analysen: Hinweise auf die verwendeten Datenquellen und die Art der Analysen.
% 5. Aufbau der Arbeit
%  Kapitelübersicht: Kurzbeschreibung der Struktur der Arbeit, um dem Leser einen Überblick zu geben, wie die Arbeit organisiert ist.
%  Kapitelinhalte: Kurze Zusammenfassungen dessen, was in den einzelnen Kapiteln behandelt wird.
% 6. Abgrenzungen
%  Eingrenzung des Themas: Festlegung der Grenzen der Untersuchung, um den Umfang der Arbeit klarzustellen.
%  Einschränkungen und Annahmen: Diskussion über die Einschränkungen der Arbeit und die Annahmen, die gemacht wurden.
