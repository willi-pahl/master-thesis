\begin{tcolorbox}[
	enhanced,
	colback=red!5!white,
	colframe=red!75!black!50,
	title= Mein roter Faden
	]
Dieses Kapitel wird die positiven und negativen Erfahrungen der Kapitel Implementierung und Evaluation auffassen. Die weiteren Kapitel bauen auf die hier gewonnenen Erkenntnisse auf.
\end{tcolorbox}


\section{Erweiterte Codeevaluation}
Bei den vordefinierten Prüfungen der HumanEval Benchmarks, wird geprüft, ob der Code lauffähig ist, nicht aber die Codestruktur oder Kommentare. Ein Problem bei der Nutzung des von der LLM generiertem Code ist, dass Entwickler diesen einfach kopieren und in ihre Programme implementieren. Es wird also nur die Funktionalität des Codes geprüft, nicht aber Strukturen und Kommentare um die Lesbarkeit und Verständlichkeit zu erhöhen. Dieses Vorgehen mag zu schnellen Erfolgen in der Programmentwicklung führen, wird aber beim Refactoring oder Fehlersuche erhebliche Defizite mit sich bringen.\vspace{0.2cm}

Aus diesem Grund sollte der erstellte Code nicht nur auf die Funktionalität geprüft werden. Dafür sollten weitere Test-Frameworks der jeweiligen Programmiersprache zur Anwendung kommen. Es gibt mehrere Frameworks zur Prüfung der Codequalität unter PHP. Zwei bekannte Frameworks die auch in dieser Arbeit Anwendung finden, sind die Frameworks \texttt{phpunit} und \texttt{phpmetrics}. Mit ihnen wird der, durch die LLMs generierten Codes geprüft.\vspace{0.2cm}

Um PHPUnit und PHPMetrics für die Evaluierung zu verwenden, müssen weitere Angaben und Einträge im Benchmark erfolgen. So muss ein PHP-Unittest enthalten sein, dieser kann den einfachen benutzerdefinierten Test ersetzen. Des Weiteren sind die Kriterien für die Metrik Messung, für jeden Test erforderlich. Die Kriterien können wie in Listing \ref{lst:phpmetric_criteria_example} dargestellt, aussehen.

\begin{lstlisting}[language=python,caption={Beispiel für Bewertungskriterien},label=lst:phpmetric_criteria_example]
	criteria = {
		"Lines of code": lambda x: int(x) > 12,
		"Logical lines of code by method": lambda x: float(x) > 7,
		"Lack of cohesion of methods": lambda x: float(x) > 3,
		"Average Cyclomatic complexity by class": lambda x: float(x) > 10,
		"Average Weighted method count by class": lambda x: float(x) > 20,
		"Average bugs by class": lambda x: float(x) > 0.1,
		"Critical": lambda x: int(x) > 0,
		"Error": lambda x: int(x) > 0,
		"Warning": lambda x: int(x) > 0,
		"Information": lambda x: int(x) > 0,
	}
\end{lstlisting}

Mit den erweiterten Tests werden die Benchmarks, um die folgenden Punkte erweitert.

\begin{myitemize}
	\item \textbf{unittest}: Unittests für die geforderte Funktion, unterschied zu den einfachen Tests
	\item \textbf{metrics}: Kriterien für den Metriktest
\end{myitemize}

\subsection{PHPUnit}
Eines der bekanntesten spezielles Framework für Unit-Tests in PHP, was als Industriestandard gilt. Mit diesem Framework können neben der Prüfung auf funktionsfähigen Code auch Randfälle betrachtet und Fehlerbehandlungen im Code getestet werden. Als Grundlage für die Auswahl des Tools wird auf Studie \cite{mohamad-2016} verwiesen.

\subsection{PHPMetrics}
Ein PHP Framework für die Codeanalyse, welches detaillierte Berichte über die Codequalität, Komplexität des Codes und über dessen Wartbarkeit erzeugt. PHPMetrics wird in verschiedenen Arbeiten eingesetzt, um die Codequalität zu ermitteln. So auch in \cite{anggrain-2016}, bei der verschiedene Open Source LMS verglichen werden.

\subsection{SonarQube}
Als letztes Tool soll SonarQube zur statischen Codeanalyse und Codeprüfung zum Einsatz kommen. Es werden verschiedene Programmiersprachen unterstützt, darunter auch PHP und JavaScript. In der Arbeit \cite{da-silva-simoes-2024} wird die Prüfung der Codequalität mit SonarQube, ChatGPT3.5 und ChatGPT4 vergleichen. Als Schlussfolgerung aus dem Ergebnis dieser Arbeit, wird auch hier die Codeanalyse durch eine LLM nicht erfolgen, sondern ebenfalls durch SonarQube.

%\subsection{ESLint}
%JavaScript Tool zur Syntaxfehler-Erkennung, Stil- und Codequalitätsprüfung. Mit diesem Tool kann reines JavaScript als auch Node.js überprüfen. https://arxiv.org/html/2402.14261v1
