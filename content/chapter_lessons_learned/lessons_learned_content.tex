In diesem werden die Arbeitsprozesse reflektiert und Fehler die während der Arbeit auftraten analysiert. Des Weiteren werden Vorschläge gegeben, um das Arbeiten mit Sprachmodellen zu verbessern. Zuerst wird der Evaluierungsprozess betrachtet, um anschließend die Optimierung der Prompts zu diskutieren.

% --- Evaluierung ----------------------------------------------------------------------------------


\section{Evaluierung der großen Sprachmodelle}
Ein großes Problem stellte der Zugriff auf die Cloused-Source-Modelle dar. Durch die beschränkten Bezahlmethoden konnte ein permanenter Zugriff auf die Modelle nicht erfolgen. Aus diesem Grund wurden hauptsächlich Open-Source-Modelle lokal evaluiert und getestet.


\subsection{Lokale Ressourcen}
Eines der größten Probleme für das lokale Betreiben von großen Sprachmodellen sind die Hardwareanforderungen. Hier spielen neben der Prozessoranzahl, die Speicherplatz der Festplatte und der VRAM der Grafikkarte eine Rolle. Während der Arbeit wurde eine SSD-Festplatte mit höherer Kapazität eingesetzt und eine Grafikkarte mit mehr VRAM.\vspace{0.2cm}

Der größere Speicher wurde notwendig während des Laden und Speichern der Modelle auf die lokale SSD. Zudem war ein RAM notwendig, der doppelt so groß sein musste wie das Modell selbst. Um diesen RAM bereit zustellen wurde eine SWAP Partition von 100 GB auf der SSD eingerichtet. Dieser wurde auch für die Ausführung größerer Modelle benötigt.\vspace{0.2cm}

Eine weitere Verbesserung war der Austausch der Grafikkarte. Hierbei wurde die vorhandene Nvidia GTX 1050 TI mit 4 GB VRAM und einer Bandbreite von 112 GB/s durch eine Nvidia RTX 3060 mit 12 GB VRAM und einer Bandbreite von 360 GB/s ersetzt. Durch das Austauschen der Grafikkarte wurde der SWAP für die Berechnungen der Token während die Antwort erstellt wurde nicht mehr benötigt.

Durch diese Anpassungen bestand die Möglichkeit größere Modelle zuladen und bereit zustellt. Des Weiteren konnte eine wesentliche Verbesserung der Antwortzeit festgestellt werden. Eine genaue Messung wurde hier nicht durchgeführt. So wurden bei dem Deekseek-Coder-V2 Modelle eine Verbesserung der Berechnungszeit für den Benchmark von etwa 24 Stunden auf circa eine Stunde beobachtet.\vspace{0.2cm}

Dennoch war die Berechnungszeit einiger Modelle, welche mehr als 12 GB groß waren sehr hoch. Sodass die Wartezeit, das Auswerten der Evaluierung verzögerte.


\subsection{Auswertung des Benchmarks}
Ein Problem bei der Generierung der Codes war der zu gering gewählte Wert für die zulässigen Tokens. Bei einer Größe von nur 600 Token, wurden die Antworten bei den Modellen \textit{ChatGPT 4}, \textit{Gemini 1.5} und \textit{Deepseek-R1} abgeschnitten, sodass die generierten Codes nicht ausführbar waren. Erst der Verzicht einer vorgeschriebene Tokenanzahl brachte lauffähige Codes hervor. Allerdings waren die Kosten beim \textit{ChatGPT 4} Modell so hoch, dass aus weitere Tests mit diesem Modell verzichtet wurde.\vspace{0.2cm}

Bei der Auswertung des Benchmarks traten einige Fehler auf, die beseitigt werden konnten. Ein Großteil waren kleinere Fehler, die bei der Auswertung aufgetreten, wie in Kapitel \ref{subsec:disadvantages_of_evaluation} bereits angesprochen. Der gravierendste Fehler trat aber bei der Berechnung das \texttt{pass@k} für das gesamte Modell auf. Hier wurde eine falsche Python-Methode implementiert, sodass das Ergebnis, um ein Vielfaches niedriger war, als das wirkliche Ergebnis. Erst durch den Vergleich mit Ergebnisse anderer Arbeiten und den Herstellerangaben ist der Fehler aufgefallen. Nach intensiver Suche wurde dieser gefunden und beseitigt.\vspace{0.2cm}

Bewehrt hat sich hier der Einsatz von Python und dessen Bibliotheken für Umsetzung der Evaluierungsaufgaben. Mittlerweile existieren für die meisten Probleme und Anforderungen bereits fertige Bibliotheken oft von den Herstellern der Modelle selbst. Basierend auf den vorhandenen Bibliotheken, konnte die Entwicklung der Evaluationsaufgaben in kurzer Zeit umgesetzt und implementiert werden.\vspace{0.2cm}

% --- Evalierung mit eigenen ---------------------------------

\begin{tcolorbox}[
	enhanced,
	colback=red!5!white,
	colframe=red!75!black!50,
	title= Mein roter Faden
	]
	Hier folgen noch die Ergebnisse zur Optimierung.
\end{tcolorbox}

% --- Optimierung ----------------------------------------------------------------------------------


\section{Optimierung der Abfragen}


\subsection{Erweiterte Codeevaluation}
Bei den vordefinierten Prüfungen der HumanEval Benchmarks, wird geprüft, ob der Code lauffähig ist, nicht aber die Codestruktur oder Kommentare. Ein Problem bei der Nutzung des von der LLM generiertem Code ist, dass Entwickler diesen einfach kopieren und in ihre Programme implementieren. Es wird also nur die Funktionalität des Codes geprüft, nicht aber Strukturen und Kommentare um die Lesbarkeit und Verständlichkeit zu erhöhen. Dieses Vorgehen mag zu schnellen Erfolgen in der Programmentwicklung führen, wird aber beim Refactoring oder Fehlersuche erhebliche Defizite mit sich bringen.\vspace{0.2cm}

Aus diesem Grund sollte der erstellte Code nicht nur auf die Funktionalität geprüft werden. Dafür sollten weitere Test-Frameworks der jeweiligen Programmiersprache zur Anwendung kommen. Es gibt mehrere Frameworks zur Prüfung der Codequalität unter PHP. Zwei bekannte Frameworks die auch in dieser Arbeit Anwendung finden, sind die Frameworks \texttt{phpunit} und \texttt{phpmetrics}. Mit ihnen wird der, durch die LLMs generierten Codes geprüft.\vspace{0.2cm}
Um PHPUnit und PHPMetrics für die Evaluierung zu verwenden, müssen weitere Angaben und Einträge im Benchmark erfolgen. So muss ein PHP-Unittest enthalten sein, dieser kann den einfachen benutzerdefinierten Test ersetzen. Des Weiteren sind die Kriterien für die Metrik Messung, für jeden Test erforderlich. Die Kriterien können wie in Listing \ref{lst:phpmetric_criteria_example} dargestellt, aussehen.

\begin{lstlisting}[language=python,caption={Beispiel für Bewertungskriterien},label=lst:phpmetric_criteria_example]
	criteria = {
		"Lines of code": lambda x: int(x) > 12,
		"Logical lines of code by method": lambda x: float(x) > 7,
		"Lack of cohesion of methods": lambda x: float(x) > 3,
		"Average Cyclomatic complexity by class": lambda x: float(x) > 10,
		"Average Weighted method count by class": lambda x: float(x) > 20,
		"Average bugs by class": lambda x: float(x) > 0.1,
		"Critical": lambda x: int(x) > 0,
		"Error": lambda x: int(x) > 0,
		"Warning": lambda x: int(x) > 0,
		"Information": lambda x: int(x) > 0,
	}
\end{lstlisting}

Mit den erweiterten Tests werden die Benchmarks, um die folgenden Punkte erweitert.

\begin{myitemize}
	\item \textbf{unittest}: Unittests für die geforderte Funktion, unterschied zu den einfachen Tests
	\item \textbf{metrics}: Kriterien für den Metriktest
\end{myitemize}

\subsection{PHPUnit}
Eines der bekanntesten spezielles Framework für Unit-Tests in PHP, was als Industriestandard gilt. Mit diesem Framework können neben der Prüfung auf funktionsfähigen Code auch Randfälle betrachtet und Fehlerbehandlungen im Code getestet werden. Als Grundlage für die Auswahl des Tools wird auf Studie \cite{mohamad-2016} verwiesen.

\subsection{PHPMetrics}
Ein PHP Framework für die Codeanalyse, welches detaillierte Berichte über die Codequalität, Komplexität des Codes und über dessen Wartbarkeit erzeugt. PHPMetrics wird in verschiedenen Arbeiten eingesetzt, um die Codequalität zu ermitteln. So auch in \cite{anggrain-2016}, bei der verschiedene Open Source LMS verglichen werden.

%\subsection{SonarQube}
%Als letztes Tool soll SonarQube zur statischen Codeanalyse und Codeprüfung zum Einsatz kommen. Es werden verschiedene Programmiersprachen unterstützt, darunter auch PHP und JavaScript. In der Arbeit \cite{da-silva-simoes-2024} wird die Prüfung der Codequalität mit SonarQube, ChatGPT3.5 und ChatGPT4 vergleichen. Als Schlussfolgerung aus dem Ergebnis dieser Arbeit, wird auch hier die Codeanalyse durch eine LLM nicht erfolgen, sondern ebenfalls durch SonarQube.

%\subsection{ESLint}
%JavaScript Tool zur Syntaxfehler-Erkennung, Stil- und Codequalitätsprüfung. Mit diesem Tool kann reines JavaScript als auch Node.js überprüfen. https://arxiv.org/html/2402.14261v1
