%\section{Large Language Models}

%https://bop.unibe.ch/iw/article/view/11053/13941
Mit der Einführung von ChatGPT für die breite Öffentlichkeit, am 30. November 2022, wurde ein Hype um die großen Sprachmodelle ausgelöst, der von da an als treiben Kraft, hinter ihrer Entwicklung gesehen werden kann. In einigen Artikel ist sogar die Rede von einer,

\epigraph[
	author={Siegfried Handschuh, entnommen aus \cite{handschuh-2024}},
	source={Große Sprachmodelle},
	text indent=0.5cm
]{
	Zeitenwende, wie sie nur alle 30-40 Jahre erleben und wie wir sie zuletzt mit der Einführung des World Wide Web gesehen haben
}

Wo knüpfe ich an?


\section{Methoden und Ansätze}


\section{Forschungslücken und zukünftige Forschung}
Künftige Forschung.


\subsection{Identifikation von Forschungslücken}


\subsection{Zukünftige Forschungsrichtungen}

