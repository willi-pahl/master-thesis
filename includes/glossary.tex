\newglossaryentry{collaborative}
{
	name=kollaborativ,
	description={Kollaborativ lässt sich auf das lateinische Wort \textit{collaborare} für zusammenarbeiten zurückzuführen. In einem kooperativen Multi-Agenten-System arbeiten die Agenten zusammen, um ein gemeinsames Ziel zu erreichen. Der Erfolg des Systems hängt davon ab, wie gut die Agenten zusammenarbeiten und Ressourcen teilen}
}

\newglossaryentry{competitive}
{
	name=kompetitiv,
	description={In einem kompetitiven Multi-Agenten-System hingegen arbeiten die Agenten gegeneinander, und ihre Handlungen sind oft darauf ausgelegt, ihre eigenen Vorteile auf Kosten anderer Agenten zu maximieren}
}

\newglossaryentry{gated_recurrent_unit}
{
	name=Gradientenabstiegsverfahren,
	description={Das Gradientenabstiegsverfahren (eng. Gated Recurrent Unit) ist ein 2014 eingeführtes Verfahren, um Optimierungsprobleme zu lösen. Von einem Startpunkt aus wird sich in Richtung des steilsten Abstieges bewegt. Ist ein Minimum erreicht, welcher mit einem Näherungswert übereinstimmt, ist das Verfahren abgeschlossen}
}

\newglossaryentry{bias}{
	name=Bias,
	description={In der menschlichen Wahrnehmung bezeichnet Bias, eine kognitive Verzerrung oder Voreingenommenheit, die beispielsweise durch ein Vorurteil zustande kommt. In einem Neuron wird der Output ebenfalls durch dessen Wert verzerrt. Künstliche Neuronen sind mit dem Bias flexibler und erlaubt eine Verschiebung der Aktivierungsfunktion und es ist ein Output möglich, auch wenn keine Inputsignale ankommen}
}

\newglossaryentry{overfitting}{
	name=Overfitting,
	description={Overfitting ist ...}
}